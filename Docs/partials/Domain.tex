\section{Domain}

Near Field Communication (NFC) technology has helped automate many functions in day-to-day life. It is an extension of the already present Radio Frequency Identification (RFID) which is used partly within the Internet-of-Things domain. NFC is used within a much closer proximity (within 10 cm) compared to its predecessor RFID which has an effective range of roughly 100 cm~\cite{9213758}. This shorter range means that certain functions can branch away from RFID and use NFC as its interface. Examples where the NFC interface is currently in use are in bus passes, credit cards, debit cards, ID badges, and even cellphones. The use of NFC in these devices are mainly used for payment systems or as a means to authorize user actions. While these devices are more convenient to use than the conventional methods, the current concern behind NFC technology is that it presents a vector for multiple types of attacks.\\

NFC Security is an important technology in payment systems, physical access security systems as well as identification. NFC payment attacks are the most common method for attackers as they are becoming easier to replicate by malicious players. A different approach is through a skimming attack which can be achieved by placing a “skimmer” on top of a registered payment system that collects information at the same time as the payment system. After skimming the information, it is stored until the malicious person can take the skimmer off and extract the data. Another kind of skimmer works by using a large antenna to grab NFC tags on physical credit cards. After receiving the data from a skimming device, malicious users can then perform a replay attack that emulates an NFC tag that contains real credit card information. The NFC Tag found in credit cards is one of the most common ways to steal credit card information as it is a passive device that is not able to easily identify when it is being read from.
